\documentclass{article}

\usepackage{fancyhdr}

\begin{document}
\pagestyle{fancy}

\fancyhead[R]{Frederik Hantho\\Jan Wiberg\\Rune Bromer}
\fancyhead[L]{Synopsis\\pupyMPI\\27. August 2009}
\setlength\headheight{2cm}

\noindent
\textbf{Project title:} pupyMPI - Pure Python MPI\\

\noindent
\textbf{Authors:} Frederik Hantho, Jan Wiberg \& Rune Bromer\\

\noindent
\textbf{Project consultant:} Brian Vinter\\

\noindent
\textbf{ECTS:} 15\\


\section*{Background}
The demand for easy, highly scalable, parallel computing is growing daily as
many-core systems become prevalent. The Message Passing Interface (MPI) is a
widely used and globally accepted model for expressing parallelism. The success
of MPI stems from its portability, completeness and performance i.r.t. massively
parallel programs, amongst others. However, MPI is generally not considered
simple, nor easy to program, particularly as it is most commonly implemented in
C or FORTRAN. This is a problem since many potential users are not skilled
programmers, but rather researchers who are capable of expressing their
algorithms in a deterministic logical framework.
\\

The difficulty involved in ensuring the correctness and maintainability of C has
lead to the rise of higher level languages, such as Python. Python is a
language that originated in academia, and has large and growing popularity in
this group where it is often used for prototyping solutions to scientific
problems. Amongst Pythons many advantages are a strong typing system, a highly
readable syntax, existence on many platforms, and it is quite mature.
\\

Python has no built-in support for MPI, but several other projects have grafted
MPI functionality onto Python, with varying degrees of completeness. At present,
there seems to be no native Python MPI implementations openly available.

%1 mpi is a globally accepted standard
%2 python is mature and widely used in scientific computing circles.
%3 python is cross platform
%4 many existing implementations in C and fortran, other languages often wrappers.
%5 both mpi and python widely used in academia and education.

\section*{Motivation}
Coupling Python with MPI would seem to have several distinct advantages. 
\begin{enumerate}
	\item The advanced type system alleviates the complexities of MPI types as
Python types can be easily inferred at runtime. 
	\item Python is standardized across platforms and is either already
installed, or very easy to install, on almost any major platform of interest to
HPC users. Python is also highly modular and easy to extend.
	\item Implementing MPI in native Python means that there is no need to
maintain a separate native MPI installation, and by extension no need to keep
specific versions of Python, the MPI wrapper and the MPI library around to
maintain binary compatibility.
	\item The low bar to entry to Python means that it is highly useful as a
prototyping language, and educators can focus on teaching central concepts of
concurrency, without the clutter of boilerplate code so typical to e.g. C.
\end{enumerate}

%1 ease of installation
%1.5 crossplatform.
%2 problems with current wrappers
%3 python highly accessible to non computer scientist researchers.
%4 education.
%5 MPI types easily handled in python.


\section*{Learning goals}
At the end of the project, we will be able to: 

\begin{itemize}
	\item{ Analyze the shortcomings of the MPI libraries written in a strict programming language. }
	\item{ Reason about the advantages and disadvantages of a native Python MPI compared with other Python MPI solutions. }
	\item{ Reason about different internal communication strategies for achieving high performance. }
	\item{ Choose a design for a high performance MPI-like library, and argue for the choice. }
	\item{ Reflect on the performance and reliability of a prototype implementation of our design. }
%	\item{ Analyze and compare different MPI libraries in terms of performance and liability }
\end{itemize}

\end{document}
